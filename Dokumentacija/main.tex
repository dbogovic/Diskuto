\documentclass{foi}
\usepackage[utf8]{inputenc}
\usepackage{lipsum}

\vrstaRada{\diplomski} % \diplomski
\title{Razvoj polustrukturirane baze podataka za potrebe internetskog foruma kao web-aplikacije}

\author{Dario Bogović}
\spolStudenta{\musko} % \zensko ili \musko
\mentor{Bogdan Okreša Đurić}
\spolMentora{\musko} % \zensko ili \musko
\godina{2020}
\mjesec{rujan}
\date{2020}
%\status{redoviti}
\indeks{47165}
\smjer{Informacijsko i programsko inženjerstvo} % (ili Poslovni sustavi, Ekonomika poduzetništva, Primjena informacijske tehnologije u poslovanju, Informacijsko i programsko inženjerstvo, Baze podataka i baze znanja, Organizacija poslovnih sustava, Informatika u obrazovanju)
\titulaProfesora{Dr. sc.}

\sazetak{Rad se sastoji od teorijskog i praktičnog dijela. U teorijskom dijelu opisan je polustrukturirani model podataka, definirane razlike u odnosu na strukturirane i nestrukturirane podatke te predstavljeni tipovi polustrukturiranih podataka, s naglaskom na XML. Slijedi opis i definiranje XML baze podataka te pregled sustava za upravljanje XML bazom podataka. Praktični dio rada sadrži prikaz rada s noSQL dokumentnom bazom podataka i programskom platformom eXist-db, uz izradu konkretne XML baze podataka. Načini pristupa XML bazi podataka, slanje upita i dohvaćanje podataka prikazani su na primjeru Java web-aplikacije za internetski forum.}

\kljucneRijeci{xml; polustrukturirani podaci; XML baza podataka; eXist-db}

\begin{document}

\maketitle

\tableofcontents

\pagestyle{plain}
\chapter{Uvod}

Obavezno napiši zašto? Zašto je ovo dobro koristiti? Zašto baš s time itd.?

Završni ili diplomski rad studenta/studentice je konačni rezultat uloženog napora u završetak studija. Obranom završnog ili diplomskog rada student/studentica stječe prava i obveze koje proizlaze iz završetka akademskog obrazovanja. S ciljem osiguranja potpore studentima pri pisanju završnog/diplomskog rada, izrađen je ovaj predložak oblikovanja samog rada.

Načelna napomena o strukturi rada jest da se nazivi i struktura poglavlja obavezno definiraju u dogovoru s mentorom/mentoricom. Sadržajna preporuka je da u uvodu treba opisati što je tema završnog/diplomskog rada, zašto je tema značajna te koja je motivacija studenta/ studentice za odabir teme. 

\chapter{Podaci}

U ovom poglavlju treba opisati koje će metode i tehnike biti korištene pri razradi teme, kako su provedene istraživačke aktivnosti, koji su programski alati ili aplikacije korišteni.

\section{Strukturirani podaci}

\lipsum[1]

\section{Nestrukturirani podaci}

\lipsum[1]

\section{Polustruktirani podaci}

\lipsum[1]

\chapter{XML}

Ovo je glavni dio rada u kojem treba razraditi temu, pojasniti istraživanja, prikazati rezultate i slično. Poželjno je na početku poglavlja dati kratki opis strukture poglavlja, kako bi čitatelj/čitateljica rada mogao/mogla lakše pratiti složenu cjelinu.

\chapter{XML baza podataka}

Tehničke upute u nastavku opisuju način tehničkog oblikovanja rada i navođenja literature.

\section{eXist-db}

\lipsum[1]

\chapter{Opis aplikacijske domene}

Ovdje treba sažeto rezimirati najvažnije rezultate razrade teme rada. Potrebno je sažeto opisati što je predmet rada, koje su metode, tehnike, programski alati ili aplikacije korištene u razradi rada te koje su pretpostavke dokazane, a koje opovrgnute. Sadržajno, ono što se u uvodu rada najavljuje i kasnije je obuhvaćeno u samom radu, moralo bi biti opisano u zaključnom dijelu kroz rezultate rada. 

\lipsum[1-2]

\chapter{Model baze podataka}

\lipsum[1]

\chapter{Implementacija}

\section{Kreiranje baze podataka}
\section{Rad s bazom podataka}
\section{Java web-aplikacija}

\chapter{Primjeri korištenja}
\chapter{Zaključak}

\printbibliography[title=Popis literature]
\addcontentsline{toc}{chapter}{Popis literature}

\listoffigures
\addcontentsline{toc}{chapter}{Popis slika}
 
\listoftables
\addcontentsline{toc}{chapter}{Popis tablica}

\end{document}
